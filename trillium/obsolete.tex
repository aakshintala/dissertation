% !TeX root = ../dissertation.tex
\section{obsolete shit}
CUDA API calls from applications in the guest VM are redirected by an interposer library through
a front-end driver (using Xen event channels) to a back-end driver in dom0, which in turn exercises the CUDA driver and runtime
directly. GViM proposes some memory management optimizations to avoid double buffering, redundant copy, such as using
{\tt mmap} to map guest kernel memory into the front end driver to avoid user-to-kernel copies, and using a page-directory
structure to avoid copy between VMs from front-end driver to back-end driver.~\footnote{Most of these optimizations should be obviated by
VM support for cross-VM bulk-transfer mechanisms like VMCI.}
Similarly, in vCUDA ~\cite{vCUDA}, CUDA API calls are redirected
through an interposer library to a stub in the host OS, which interacts with the device using pass through.
Cross-VM communication is implemented with XMLRPC~\cite{XMLRPC}, which turns out to be the dominant negative term in performance:
RPC is slow. The system supports a ``lazy RPC optimization''  which effectively batches RPCs.
gVirtuS~\cite{gVirtuS} is an API remoting framework that claims to provide
transparent support for CUDA, OpenCL, and OpenGL on Xen, KVM, and VMware VMs, using a front-end/back-end
to provide a formal abstraction layer for GPUs that is independent of VMM. Details are sketchy.

vmCUDA~\cite{vmCUDA} supports API remoting for CUDA in the the ESX Hypervisor.
vmCUDA employs a
standard split-driver model with a front-end driver in the guest, and a backend driver in the control domain (called
the ``appliance VM'') which interacts with the CUDA runtime and driver.
The driver in the appliance VM interacts with the GPU hardware using pass-through.
CUDA applications in the guest are linked against an interposer
library which uses vRDMA, VCMI, TCP to forward calls and data to the appliance VM. The application starts on the client,
sends a copy of binary to appliance VM, which modifies the binary and starts a process container for it: the client VM
communicates with that process. The authors find that data copy calls must be broken down into smaller fragments to enable
multiple VMs to share the GPU. Cross-VM isolation guarantees result from use of per-application child processes
in the appliance VM, which effectively level process-level protection mechanisms toward cross-VM protection.
The design addresses compatibility challenges with VM mobility (vMotion): the appliance VM needn't move if the
guest VM moves.~\footnote{This is a fairly limited form of support for mobility, and is not likely what the
user intends when migrating a VM. That said, API remoting is fundamentally resistant to mobility, and this
is the best known solution as of this writing.} vmCUDA performs dynamic binary re-writing of the client application
(replacing API calls) to make API forwarding transparent to the developer.

\subsubsection{User-mode API remoting}

rCUDA~\cite{rCUDA, rCUDAnew} is a middle-ware system for multiplexing
	NVIDIA GPUs and CUDA across a cluster.
	A client library encapsulates access to a (potentially) remote GPU. Client applications must
	be recompiled/re-linked against the rCUDA library, which results
	in feature incompatibilities for a number of undocumented features that are handled transparently
	by the {\tt nvcc} compiler. While the basic design is isomorphic to the API remoting design, virtual machines
	need not be present.
GridCuda~\cite{GridCuda} is similar: a pure
	user-mode fabric for tunnelling CUDA API traffic to GPUs on remote machines.
	SnuCL~\cite{kim2012snucl} provides an OpenCL~\cite{openCLspec} programming interface
	to clusters of CPU/GPU servers. The basic approach is to extend the OpenCL semantics to encapsulate
	remote resources, which preserves the original programming model which, like CUDA~\cite{CUDA:Programming-Guide}, assumes
	a process model, running in the context of a single operating system image. Like rCUDA~\cite{rCUDA}
	and GridCuda~\cite{GridCuda}, SnuCL should be viewed as a distributed runtime that supports GPUs,
	rather than a general approach to GPU virtualization.
 VCL~\cite{VCL} is the lower layer in a package called MGP (many-gpu-package)
   which encapsulates remote compute resources and
   data management within the local OpenCL API implementation, such that remote GPUs look to the application running on the
   master node, like a local OpenCL device.
   Similar techniques are described in~\cite{Duato:2009:EIG:1884795.1884840,Li:2011:GRS:2066302.2066933, Xiao:2012:TAM:2310096.2310143}.

\subsubsection{RPC frameworks}

gRPC, xmlrpc-c

\paragraph {\bf gVirt}. gVirt claims to be a full-virtualization but relies on mediated pass-through~\cite{gVirt}.
     gVirt represents an Intel-centric technique for implementing virtualization of \emph{graphics}, but as full-virtualization system
	 the techniques are relevant to GPGPU. The system runs the native graphics driver
	 in the guest OS in dom0, and implements pass-through for access only to performance-critical resources (command and frame buffers),
	 using trap-emulate for resources generally accessed off the critical path (PTEs, I/O Registers). The native driver is present
	 primarily to simplify tasks like initialization and power management. Trap-emulate
	 operations are forwarded to a mediator module in dom0, which implements the vGPU interface and scheduler. Operations are subsequently
	 handled with hypercalls into a stub in Xen. Memory is multiplexed with a combination of partitioning and `` ballooning''.
	 Each VM gets 2GB of local graphics memory and 2GB global graphics memory. These partitions are striped across the actual physical
	 memory, and sections not belonging to a particular VM are marked ``ballooned'' in the page tables of that VM, which means they
	 are inaccessible. This makes address translations exactly the same for each VM with the
	 caveat that regions belonging to different VMs must be made inaccessible. gVirt handles this through
	 ``smart shadowing'' and auditing of the command buffer.
	 % The primary limitations of this work are as follows. First, the
	 % solution is clearly geared toward graphics. Second, the partitioning of the memory space means the full physical memory
	 % can never be utilized by each VM. Finally the auditing process seems tenuous. You can check that register values are
	 % bounded by regions that should be mapped to a given VM, but how do you know for sure when a register value is actually
	 % a pointer? Can you cause protection faults in programs that happen to use values large enough to look like pointers?
	 % Can memory instructions with indirection and offsets be used to circumvent the audit?

\paragraph {\bf GPUvm}.
      GPUvm~\cite{GPUvm} GPUvm provides full and para-virtualization support for
	  CUDA on Kepler and Fermi (NVIDIA) GPUs for Xen.
	  and implementing a unified address space across both CPU and GPU memory.
	  GPUvm safely multiplexes the basic GPU physical resources: GPU contexts (analogous to a process), GPU channels (the mechanism
	  by which commands are submitted to a context), GPU page tables, and GPU control registers, which are memory apertures
	  mapped onto PCIe BARs with MMIO.
	  To this end, the design introduces \emph{GPU shadow
	  channels}, \emph{GPU shadow page tables}, and \emph{virtual GPU schedulers}: these abstractions
	  form the interposition boundary for virtualization.
	  GPUvm presents a native GPU device model to each VM, each of which in turn accesses the GPU through a
	  GPU Access aggregator module. The aggregator manages the GPU, maintains shadow page tables, shadow
	  channels, and enforces fairness in scheduling. The aggregator may modify requests to enforce isolation.
	  GPUvm interposes on communication between guest device driver and the GPU device model, it marks the MMIO
	  ranges inaccessible so it can then trap and forward. Like gVirt, GPUvm statically partitions resources among VMs.

	  The performance costs of full virtualization are unacceptable, and primarily result from page table
	  management overheads. TLB flushes are required with every GPU page table update. Moreover, GPU page faults
	  are not forwarded to the host CPU, so GPUvm must scan GPU page tables on every TLB flush to keep
	  GPU shadow page tables current. A number of optimizations ameliorate these overheads.
	  \emph{Lazy Shadowing} reflects guest updates to page tables into shadow page tables
	        \emph{only} when the tables are referenced, rather than on every TLB flush.
	  \emph{Bar Remap} limits BAR interposition and passes through BAR accesses other than those made to GPU channel descriptors.
	  \emph{Para-virtualization} allows a guest GPU driver updates page tables through hypercalls (which can
	  be further optimized with \emph{multi-calls}), and GPUvm
	        validates those updates, eliminating the scan of guest GPU page tables. Naturally, this optimization
			gives up full virtualization, as guest GPU drivers must be modified.


%\subsubsection{Gdev}



%however, opacity at the
% OS layer forces full virtualization to rely on memory protection techniques that
% make performance generally unacceptable, even in the presence of heavy optimization.
% Conversely, because API remoting simply forwards user-mode API calls, it is generally
% simple to implement and can provide good performance if data transfer is managed carefully.
% However the technique sacrifices interposition, forces user-mode run-times into the TCB, and is
% not robust to the evolution of the APIs being remoted: virtualization support must be updated
% with every new version of the API. Moreover, API remoting ties the system to a particular
% front end API: support for CUDA, DirectX, OpenCL and other frameworks must be implemented
% separately, which involves non-trivial engineering effort.