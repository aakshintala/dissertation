% !TeX root = ../dissertation.tex
\section{Conclusion}
\label{sec_con}

Virtualizing GPGPUs is a balancing act: there are multiple design points each offering a different trade-off of key virtualization properties. This chapter provides the first (to our knowledge) comprehensive empirical and qualitative comparison of a wide range of fundamental virtualization techniques from the literature.
We implemented GPGPU support for an SVGA-like design in the Xen hypervisor, by completing a long-missing element---the TGSI compiler---in order to leverage OpenCL support provided by the Mesa/Gallium graphics stack for Linux, via the Clover~\cite{GalliumCompute-web} project.
We proposed an improved design called \Trillium that removes the necessity for the vISA defined by SVGA resulting in dramatic performance improvements.
\Trillium represents a local optima in the GPGPU virtualization space---by decoupling device virtualization from GPU ISA virtualization, it maintains the virtualization benefits of a para-virtual system, while exhibiting the performance of a user-space remoting system. While \Trillium exhibits greater overhead than the API-remoting scheme, it is existence proof of a hypervisor-mediated API-remoting.