% -*- fill-column: 85; -*-
%!TEX root = ../dissertation.tex

%\vspace{-.1cm}
\section{Conclusion}
\label{sec:concluion}
%\vspace{-.1cm}
Virtualization techniques that rely on clean separation of software layers are
untenable for accelerator \emph{silos}.
\Model is an alternative approach that interposes compute-offload APIs,
uses automation to provide agility, recover hypervisor
interposition, and
shorten development cycles.% from years to days.


% Accelerator stacks are \emph{silos}, making virtualization techniques
% that rely on clean separation of software layers untenable. We explore
% an alternative approach that interposes framework APIs,
% uses automation to provide agility, and recovers hypervisor
% interposition and policy enforcement by managing communication transport.
% \model virtualizes compute-offload APIs by automatically constructing
% infrastructure, shortening the development cycle
% from years to days.

%\model exhibits close-to-native performance, scalability, and fair resource management across VMs.
%We hope that we have convinced the reader to embrace this radical departure from the traditional system building philosophy.


% \model exhibits close-to-native performance with integrated optimizations, as well as the good fairness and scalability.
% In this paper, we propose \model which remotes user-mode accelerator APIs through the hypervisor.
% and guarantees close-to-native performance, fairness and isolation.
% We have virtualized complete sets of OpenCL, CUDA and TensorFlow APIs,

