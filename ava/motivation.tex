% -*- fill-column: 85; -*-
%!TEX root = ../dissertation.tex


\subsection{Motivation}
\label{s:motivation}

For accelerator silos, the \emph{only} stable and
efficiently interposable interface is the framework API, so
we focus on techniques to recover or compensate virtualization properties lost by
API remoting: interposition and compatibility.
Interposition can be recovered by using hypervisor-managed forwarding transport,
creating a central interface at which to enforce resource management policies.
\Model uses a novel technique called \emph{\noveltechnique (\novtechabbrv)} to achieve this.
\novtechabbrv presents guest VMs with an abstract virtual device with MMIO Base Address Registers (BAR), but this device is \emph{not a virtual accelerator}, but an endpoint that
routes communication through the hypervisor.

Using hypervisor-managed transport recovers interposition, but complicates compatibility and introduces engineering effort:
\novtechabbrv requires custom guest libraries, guest drivers, and API servers for each OS and API, and API-specific resource-management code in the hypervisor.
\model mitigates this with automated construction (\S\ref{s:api}).
Automatically generating code to implement \novtechabbrv components presents several challenges which
follow from the need to specify API semantics and policies for which
existing Interface Description Languages (IDLs)~\cite{Lamb1987,MSIDL} are not applicable.
\Model uses a new DSL called \speclang, a compiler called \compiler, and device-agnostic transport components to address these challenges.

% because semantic information %and resource management
%specifications required for virtualization are beyond the scope of existing IDLs.
%sharing policy specifications that are beyond the scope of IDLs.
%Designing efficient and flexible transport layers, and enforcing resource-sharing policies at the transport layer where semantic %information about the underlying device resources is not readily available are additional challenges.
% In subsequent sections, we show how
