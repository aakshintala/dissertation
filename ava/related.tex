% -*- fill-column: 85; -*-
%!TEX root = ../dissertation.tex
\section{Related Work}
\label{s:related}

Chapter~\ref{sec:related} provides detailed related work; this section
addresses related work not covered there that are closely related to this
chapter.

\paragraphbe{FPGA virtualization} has a long history~\cite{codezero,
plessl2005zippy, score,tartan06asplos,virtualRC,huang09fpgavirt,brant2012zuma,
rcmw, intermediate-fabrics}. Most prior work relies on hardware-specific
features, focuses on sharing in a single protection domain~\cite{amorphos}, or
virtualization primitives~\cite{cascade}. \hira can be combined with any of
these techniques to virtualize FPGA accelerators. Our implementation
virtualizes a bitcoin workload on an FPGA with AmorphOS~\cite{amorphos}.

\paragraphbe{Nooks}~\cite{nooks} uses kernel-level interposition mechanisms
that are similar in spirit to \AvA. \AvA's compiler generates components that,
like the wrappers and XPC in Nooks, provide transparent control across address
space and machine boundaries. Object tracking and shadow copies in Nooks'
\emph{NIM} are similar to the object tracking and shadow buffers in \AvA.

\paragraphbe{RPC frameworks}~\cite{grpc,thrift,corba,Yang1996} provide an
interface description language (IDL) and tools to easily implement those
interfaces. Unlike \lapis, these IDLs do not capture all the semantics of
\emph{existing} C interfaces required to implement a \hira API remoting design.
\CAvA also generates code for controlling remote resources.

\paragraphbe{Program specification languages}~\cite{mssal} allow programmers
to specify properties of functions and their behavior, and are generally used
to check correctness, either statically (e.g., with model checking) or
dynamically (e.g., by inserting checks in the program). While such languages
allow (nearly) arbitrary predicates on programs, they are not designed to
provide semantic information to other tools, In addition, these languages are
not designed to specify how API calls are performed, and do not support
features like state tracking. \Lapis is optimized to allow easy and
specific descriptions of APIs and how calls should be performed.

\paragraphbe{Foreign Function Interface} tools allow one language to call
functions written in another, such as C. Some~\cite{swig} make use of C
headers, but require manual annotations in many common cases. Unlike \AvA,
language specific DSLs~\cite{cython,jna}, do not support marshalling data
structures and encapsulate, rather than export, the C API. Cross-language
serialization standards and frameworks~\cite{protobuf, msgpack, ecma404}, only
provide serialization for a set of primitives and supported constructs. The
user must write code to translate their data structure into the language of
the framework and provide their own transport for the serialized data.