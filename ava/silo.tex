% -*- fill-column: 85; -*-
%!TEX root = ../dissertation.tex
\section{Accelerator Silos}

Domain Specific Accelerator software stacks are effectively \emph{silos},
interposable only at the top (user-space API) and at the bottom (hardware
support for virtualization). Intermediate layers of DSA software stacks
\emph{cannot be practically separated} to virtualize the device.
The software stacks of Domain Specific Accelerators are typically composed of
a library that exposes a user-space API, and a runtime and a kernel driver
that manage and orchestrate computation on the DSA. These components typically
communicate via explicitly unstable (and often undocumented) interfaces and
protocols, which enables the vendor to preserve forward compatibility easily.
Where possible, DSA software also typically uses kernel-bypass communication
techniques to eliminate overheads from the kernel. Scheduling and resource
allocation for DSAs are typically under the control of a combination of the
proprietary CPU-based runtime (user-space and kernel), and the controller on
the DSA itself, not the CPU-based Operating System. DSAs typically contain an
entire software stack on the hardware module, which is hidden from the
programmer. This software is used to implement a command-based programming
interface that enables the vendor’s CPU-based control software for scheduling
and resource allocation. End users work with the DSA almost entirely in higher
level languages, typically the language the DSL is embedded in (e.g., Python
for Google’s TPU, C/C++ for Nvidia and AMD GPGPUs). The DSL/API is supported
by a compiler (which converts the user’s program to the the ISA of the
computation units on the DSA), a user-space runtime and a kernel module (which
work together to implement the user-space API, and control the DSA).

Chapter~\ref{chapter:trillium} showed that virtualization
designs that interpose these opaque and frequently-changing interfaces are
\emph{impractical}: these design depend on inefficient interposition
techniques and yield solutions that sacrifice \emph{compatibility}.

The user-space API is the \emph{only} stable and efficiently interposable
software interface for DSA \textit{silos}. From our analysis of the design
space using the \iemts framework in Chapter~\ref{chapter:iemts}, we also
identified an opportunity to recover or compensate for interposition and
compatibility, virtualization properties that are traditionally sacrificed by
user-space API remoting. Interposition can be recovered by using a
hypervisor-managed transport, instead of a user-space RPC channel. The
hypervisor-managed channel also represents a central interface at which to
enforce resource management policies.