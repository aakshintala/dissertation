% !TeX root = dissertation.tex
\chapter{IEMTS --- A new accelerator virtualization taxonomy}
\label{sec:iemts}

Traditionally, virtualization designs have been taxonomized
according to the core techniques employed (e.g. emulation, full- or
para-virtualization, API remoting, etc.), and evaluated
in a property trade-off space comprising performance,
compatibility, interposition, and isolation. \emph{Isolation} ensures that
mutually distrustful guests cannot access each other's data or harm each
other's performance. \emph{Compatibility}, characterizes how well a design
preserves the freedom of hardware and software components to evolve
independently: e.g. changes in the hypervisor should not force changes to
guest software. Virtualization provides an indirection layer between
logical and physical resources by \emph{interposing} a well-defined interface.
The quality of interposition determines the nature of benefits (e.g. extent of
consolidation) afforded by a virtualized system~\cite{waldspurger12cacm}.

Virtualization techniques are well explored, yielding conventional
wisdom about their fundamental trade-offs. For example,
\emph{full virtualization} interposes the software-hardware interface
to provide a virtual view of the underlying hardware. This enables guests to
run unmodified OS and application binaries, yielding high compatibility.
However, hardware interfaces for GPUs rely heavily on MMIO and communication
through memory, which necessitates page-fault-based interposition~\cite{
tian2014full, intel_kvmgt,kindratenko2009gpu,montella2012general} techniques
that cripple performance.
\emph{Para-virtual} designs export an abstract device to the guest,
but require hypervisor-specific drivers and runtime libraries in the
guest, trading compatibility for improved performance.
\emph{API remoting (or forwarding)}~\cite{gupta2009gvim, dowty2009gpu,
giunta2010gpgpu, shi2012vcuda} aggregates high-level API calls
issued in VMs, running them on the host or in a dedicated appliance VM.
This technique can provide near native performance because API calls
are infrequent, but has poor compatibility because it requires
changes in guest applications or libraries.

\begin{table*}[tt!]
\centering
\footnotesize
\resizebox{\textwidth}{!}{%
\begin{tabular}{@{}p{0.25\linewidth}|p{0.18\linewidth}|p{0.18\linewidth}|p{0.2\linewidth}|p{0.2\linewidth}@{}}
\toprule
\multirow{2}{*}{}         & \multirow{2}{*}{GPUvm} & \multicolumn{2}{c|}{VMware SVGA}         & \multirow{2}{*}{rCUDA} \\
                          &                        & Control Interface  & GPU ABI             &                        \\ \midrule
Interposed Interface      & MMIO/BAR               & DirectX APIs       & Device ISA          & Userspace API          \\
Interposition Source      & Trap handler           & Guest driver/libs  & Guest Driver        & Guest Library          \\
Interposition Destination & Host driver            & Host framework     & Host Driver         & Host/Server Daemon     \\
Interposition Mechanism   & Trap                   & Guest library      & Compilation to vISA & Guest Library Shim     \\
Transport                 & Fault                  & Hypervisor FIFOs   & Hypervisor FIFOs    & RPC                    \\
Synthesis                 & Emulation              & Call host API      & Binary translation  & Call Server API        \\ \bottomrule
\end{tabular}
}
\caption{Comparing virtualization designs using the \texttt{IEMTS} framework.}
\label{tab:new-dims}
\end{table*}

We argue that the current \emph{de facto} taxonomy and property trade-off space
are illustrative but not informative for GPUs:
there is a large body of research that has had little influence on practice.
First, Classifying virtualization designs as API-remoting vs. full vs.
para-virtual captures important concepts, and emergent properties compactly,
but doesn't explain their correlation to properties like performance.
Second, virtualization properties such as compatibility, isolation, and
interposition have highly context-dependent meaning and their relative value
to system designers can be hard to quantify.
Consider compatibility: there are many dimensions to compatibility (library,
hardware, OS, etc.), and each of those are commonly achieved by separate
technical, and non-technical means (e.g., TGSI is the common vISA for both the
VMware and GNU/Linux graphics stacks; this is \emph{not a lucky coincidence}).
It is common to see systems compared to other systems intended as exemplars
for a technique or design pattern. Ironically, the end-to-end comparison
presented in the prior chapter on \Trillium is guilty of exactly this: we
compared \Trillium against other systems intended to represent ``full
virtualization'', ''API remoting'', and so on.
Methodologically, such a comparison is laudable: evaluating a design
exhaustively against plausible alternatives in an end-to-end setting is
fundamental to good science. However, its value for drawing out fundamental
insights is scant because ``full virtualization'' only talks about the quality
of the guest-visible abstraction, while findings involving ``API remoting''
are really about the particular API in question. ``Paravirtualization'' is
often cast as a design dimension, ultimately grouping designs that share no
core techniques, such as SVGA and GPUvm. As a framework within which to seek
higher-level insights about design, a rough taxonomy that fails to cleanly
separate most concerns is in unacceptable.

We argue that practical design goals, such as providing a virtualization layer
with specific characteristics, get obscured when these properties are
considered as a set of constraints that must be preserved, without first
refining for context.
Further, production systems, such as VMware SVGA~\cite{dowty2009gpu},
compose multiple virtualization techniques in order to leverage the best
properties of each technique, especially in the presence of multiple
interfaces.

To enable a cleaner separation of concerns, we draw on the observation that
\textit{all} virtualization relies on encapsulation and interposition, and
note that a design can be clearly understood by identifying:
\begin{itemize}[nosep, topsep=0em, leftmargin=1em,labelwidth=*,align=left]
\item the \textbf{I}nterface that is interposed,
\item the \textbf{E}nd-points (source and destination) the interposed event is
transported between,
\item the \textbf{M}echanism used to interpose,
\item the \textbf{T}ransport mechanism used to communicate between endpoints,
\item the mechanisms used to \textbf{S}ynthesize or implement the desired
functionality at the destination. We call this the \textbf{\texttt{IEMTS}} framework.
\end{itemize}

Table~\ref{tab:new-dims} presents analysis of three prior GPU virtualization
systems under the \texttt{IEMTS} framework. A quick glance at the table
tells us that GPUvm's~\cite{suzuki2014gpuvm} performance woes arise from its
realiance on trapping and emulating the guest's MMIO accesses. VMware's SVGA
has two entries in the table because there are two interfaces being
virtualized: the control interface (the DirectX/OpenGL API) and the shader ISA.
Explicitly separating the two interfaces helped us realize that
ISA-virtualization is not necessary for accelerator virtualization in the
Trillium project. Further, it became obvious to us that the control interface
virtualization in SVGA and the API-remoting in rCUDA look almost the same
under IEMTS with the exception of the transport. This observation led us
to design AvA's hypervisor-mediated API-remoting scheme, and shift our
attention to solving the compatibility effort via automation.