\begin{center}
\vspace*{52pt}
{\normalfont\textbf{ABSTRACT}}
\vspace{11pt}

\begin{singlespace}
Amogh Akshintala: Toward Efficient and Realizable Virtualization of Compute Accelerators \\
(Under the direction of Donald E. Porter, and Christopher J. Rossbach)
\end{singlespace}
\end{center}

This dissertation is concerned with software techniques for fair, efficient
and isolated sharing of Domain Specific Accelerators (DSAs) (GPUs, TPUs, etc.)
that are programmed through an API (CUDA, Tensorflow, etc.).
The main takeaway from this dissertation is that unlike with CPUs, where the
ISA is the canonical interface exposed to the programmer, the vendor-provided
user-space API is the right interface to interpose in order to virtualize
DSAs. In order to support this hypothesis, we first quantify the inefficacy of
canonical techniques from prior work that interpose on other interfaces in the
DSA software stack through empirical analysis. We find that the only canonical
technique that is able to provide low-overhead virtualization (any technique
that introduces high overhead is a non-starter for DSAs; Accelerators should
accelerate, after all) interposes the user-space API, but does so in a manner
that precludes hypervisor-interposition. Precluding hypervisor interposition
results in the inability to enforce key virtualization properties like
fairness, and isolation. In order to arrive at a virtualization technique that
satisfies the needs of DSA virtualization, we develop a novel analysis
framework (\iemts). \iemts characterizes virtualization designs based on how
the \textbf{I}nterface they perform interposition, the \textbf{E}nd-points
(in the guest and the host) involved, the \textbf{M}echanism of interposition,
the \textbf{T}ransport used to connect the endpoints, and the way the
interposed functionality is \textbf{S}ynthesized in the guest. Insights from
analyzing prior techniques using the \iemts framework enable a novel
virtualization scheme specifically designed for API-controlled Domain Specific
Accelerators, \hirafull (\hira). \hira interposes the user-space API,
transports the interposed API calls to the host via a hypervisor managed
transport (thereby enabling the hypervisor to exercise control over the DSA's
resource), and then synthesizes the interposed functionality in the host by
calling the vendor framework for the given API. Empirical analysis of \AvA, a
DSA virtualization system prototyped on KVM that realizes \hira, shows that
\hira introduces low overhead while also enabling the hypervisor to enforce
fairness, and isolation.

\clearpage