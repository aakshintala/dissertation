\begin{center}
\vspace*{52pt}
{\normalfont\textbf{ABSTRACT}}
\vspace{11pt}

\begin{singlespace}
Amogh Akshintala: Toward Efficient and Realizable Virtualization of Compute Accelerators \\
(Under the direction of Donald E. Porter, and Christopher J. Rossbach)
\end{singlespace}
\end{center}

This dissertation is concerned with software techniques for fair, efficient
and isolated sharing of Domain Specific Accelerators (DSAs), such as GPUs and
TPUs, that are programmed through an API, e.g., CUDA, Tensorflow.
Specifically, we explore the following hypothesis: the vendor-provided
userspace API is akin to the ISA for CPU virtualization, and is, therefore, the
right interface to interpose to virtualize DSAs.

We first present empirical analysis of canonical virtualization techniques, in
order to understand their inefficacies. We find that most canonical techniques
introduce high performance overhead, which is a non-starter for DSA
virtualization: Accelerators should accelerate, after all.
The only canonical technique that is able to provide low-overhead
virtualization interposes the user-space API, but does so in a manner that
precludes hypervisor interposition. Precluding hypervisor interposition results
in the inability to enforce key virtualization properties like fairness and
isolation.

In order to arrive at a virtualization technique that satisfies the needs of
DSA virtualization, we develop a novel analysis framework: \iemts. \iemts
characterizes virtualization designs based on the \textbf{I}nterface
interposed, the \textbf{E}nd-points (in the guest and the host) involved in the
interposition, the \textbf{M}echanism of interposition, the \textbf{T}ransport
used to connect the endpoints, and the way the interposed functionality is
\textbf{S}ynthesized in the guest.

Insights from analyzing prior techniques using the \iemts framework enable a novel virtualization scheme specifically designed for API-controlled DSAs: \hirafull (\hira). \hira interposes the user-space API, and transports the
interposed API calls to the host via a hypervisor managed transport. Use of a
hypervisor managed channel, coupled with semantic information of the interposed
API, enables the hypervisor to exercise control over the DSA's resources. Once
the access has been checked and schedulec by the hypervisor, the interposed
functionality is synthesized in the host by calling the vendor framework for
the given API.

Empirical analysis of \AvA, a prototype realization of \hira for KVM, shows
that \hira introduces low overhead (~15\% on average) while also enabling the
hypervisor to enforce fairness and isolation. We wrap up by characterizing
adverse scenarios for \hira, and propose solutions for the common case.

\clearpage