% !TeX root = proposal.tex
\section{Hypervisor-mediated API-remoting}
\label{sec:ava}

Practical virtualization must support sharing and isolation under flexible
policy with minimal overhead. The structure of current accelerator stacks
makes this extremely difficult to achieve.
Accelerator stacks are \emph{silos} (Figure~\ref{fig:silo})
comprising proprietary layers communicating through memory mapped interfaces.
This opaque organization makes it \emph{impossible} to interpose intermediate
layers cleanly to form a virtualization boundary. Practically interposable
alternatives leave designers with a Hobson's choice between critical
virtualization properties such as interposition and compatibility.

We present \textsc{AvA}, a system that addresses the fundamental limitations
of existing accelerator virtualization techniques.
\textsc{AvA} combines API-agnostic para-virtual I/O stack components with a
Domain-Specific Language (DSL) and toolchain to automate construction
and deployment of guest libraries and API servers.
\textsc{AvA} uses an abstract para-virtual device
to serve as a transport endpoint for forwarding the public APIs of vendor-provided frameworks
(e.g. CUDA or TensorFlow).
Unlike currently popular user-space API remoting solutions~\cite{bitfusion,xaas,vmCUDA,rCUDA,cu2rcu},
\textsc{AvA} preserves hypervisor-level resource management and strong
isolation using a novel technique called \emph{{{H}ypervisor {I}nterposed {R}emote {A}cceleration} (HIRA)}.
\textsc{AvA} forwards API calls over hypervisor-managed communication channels,
inserting automatically-generated resource management components between
traditional front- and back-ends
to enforce policies described in the DSL specification.
Critically, \emph{automation} from \textsc{AvA} enables hypervisors to keep up with fast accelerator evolution: automatic generation of components minimizes engineering effort.
As Figure~\ref{fig:trends} shows, a solution that tracks API framework evolution can track hardware evolution as well.

\textsc{AvA} supports a broad range of currently-shipping
compute offload accelerators:
We virtualized ten accelerators including NVIDIA and AMD GPUs,
Google TPUs, and Intel QuickAssist.
Virtualizing an API framework using \textsc{AvA} requires modest developer effort:
a single developer virtualized OpenCL in a handful of days,
a stark contrast to the person-years of developer effort for
VMware's SVGA II or Bitfusion's FlexDirect~\cite{bitfusion}.
Experiments show that \textsc{AvA} provides
near-native performance (e.g., 2.4\% slowdown for TensorFlow and 5.6\% for CUDA),
enforces isolation and fair sharing across guests,
and supports live migration.
We make the following contributions:

\begin{itemize}[nosep,leftmargin=1em,labelwidth=*,align=left]
\item We demonstrate feasibility of automatically constructed virtual accelerator support, showing that a single technique can deal with many architectures, APIs, versions, and policies.
\item We introduce {{H}ypervisor {I}nterposed {R}emote {A}cceleration} (HIRA) to enable hypervisor-enforced isolation and sharing policies unachievable with current SR-IOV and API remoting systems.
\item We utilize a novel DSL, \textsc{Lapis}, for describing API functions, resources, and policies to enable automatic construction of virtual stacks from native header files.
\item Our evaluation shows low developer effort, strong isolation, and good performance.
\end{itemize}
