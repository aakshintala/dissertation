% !TeX root = proposal.tex
\section{IEMTS --- A new accelerator virtualization taxonomy}
\label{sec:iemts}

Traditionally, virtualization designs have been taxonomized
according to the core techniques employed (e.g. emulation, full- or para-virtualization,
API remoting, etc.), and evaluated
in a property trade-off space comprising performance,
compatibility, interposition, and isolation. \emph{Isolation} ensures that mutually distrustful
guests cannot access each other's data or harm each other's performance. \emph{Compatibility},
characterizes how well a design preserves the freedom of
hardware and software components to evolve independently: e.g. changes in the hypervisor
should not force changes to guest software.
Virtualization provides an indirection layer between
logical and physical resources by \emph{interposing} a well-defined
interface. The quality of
interposition determines the nature of benefits (e.g. extent of consolidation) afforded by a
virtualized system~\cite{waldspurger12cacm}.

Virtualization techniques are well explored, yielding conventional
wisdom about their fundamental trade-offs. For example,
\emph{full virtualization} interposes the software-hardware interface
to provide a virtual view of the
underlying hardware. This enables guests to run unmodified OS and
application binaries, yielding high compatibility. However, hardware interfaces
for GPUs rely heavily on MMIO and communication through memory, which
necessitates page-fault-based interposition~\cite{tian2014full,
intel_kvmgt,kindratenko2009gpu,montella2012general} techniques that cripple
performance.
\emph{Para-virtual} designs export an abstract device to the guest,
but require hypervisor-specific drivers and runtime libraries in the
guest, trading compatibility for improved performance.
\emph{API remoting (or forwarding)}~\cite{gupta2009gvim, dowty2009gpu,
giunta2010gpgpu, shi2012vcuda} aggregates high-level API calls
issued in VMs, running them on the host or in a dedicated appliance VM.
This technique can provide near native performance because API calls
are infrequent, but has poor compatibility because it requires
changes in guest applications or libraries.

We argue that the current \emph{de facto} taxonomy and property trade-off space
are illustrative but not informative for GPUs:
there is a large body of research that has had little influence on practice.
We suggest an alternative framework called \textsc{IEMTS} that teases apart design
axes that are implicitly and unnecessarily intertwined in much of the literature.
\textsc{IEMTS} enables clearer understanding of trade-offs in prior designs
and provides a model for comparison of alternative designs.
While we hypothesize that our findings generalize to a broad array of accelerators
we focus on GPGPUs because they are easily available and have rich research literature.

\aak{add the table and some discussion of what IEMTS actually is.}