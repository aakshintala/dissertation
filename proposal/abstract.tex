% !TeX root = proposal.tex
\begin{abstract}
The proposed dissertation will focus on developer effort and compatibility in
software virtualization of CPU ISAs, and software virtualization of
specialized compute devices (e.g., GPUs, TPUs) that are programmed through an
API.

Although binary translation is a well-established software ISA virtualization
technique, given the size and complexity of today's dominant ISAs, developers
are routinely forced to adopt ad-hoc techniques to prioritize development
effort. The proposed dissertation will present a principled approach to
determine priority among different parts of the ISA. We believe this data will
be useful to designers of virtual ISAs as well.

Specialized compute accelerators, such as GPUs and TPUs, are usually
controlled through a user-space API.
The proposed dissertation will show that unlike with CPUs, where the ISA is
the canonical interface provided to the programmer, ISA virtualization is
untenable for specialized compute accelerators. Further, the proposed
dissertation will present a present a novel taxonomy, \emph{IEMTS} for cleanly
understanding the design space for virtualizing compute accelerators.
Based on insights from this taxonomy, the proposed dissertation will
present a novel virtualization technique, \emph{hypervisor-mediated
API-remoting}, that is at once realizable and performant.
\end{abstract}