% !TeX root = proposal.tex
\section{Introduction}
\label{sec:intro}

Virtualization has a long and tumultuous history. Virtual memory was first
described by German physicist Fritz-Rudolf Güntsch in his doctoral
dissertation in 1956~\cite{virtual-memory} and commercialized~\cite{atlas-vm}
in the Cambridge University/Ferranti Inc. Atlas computer. Virtually all
computers since then support virtual memory, with most providing hardware
units---\emph{Memory Management Unit (MMU)}---to accelerate the virtualization
of memory.

\emph{Hardware virtualization}~\cite{cp40}---the idea of virtualizing the
entire computer to enable the simultaneous execution of multiple Operating
Systems (OS)---was invented in 1962, and commercialized as the IBM
VM-370~\cite{vm370} hypervisor for the IBM 370 computer.
Virtualization was briefly forgotten through the 1980s and 1990s, as the
mainframe computer became all but obsolete during the Personal Computer (PC)
revolution. Intel's x86 Instruction Set Architecture (ISA), which came to
dominate the PCs that transplanted the mainframes, was not designed to be
traditionally virtualizable~\cite{popek-goldberg}, and was widely considered
unvirtualizable~\cite{gelsinger-pc,bugnion-workstation}. Multiple vendors
introduced \emph{software emulation} based solutions to enable the execution
of one OS on top of another (e.g., Insignia SoftPC, Connectix VirtualPC,
VMware Workstation). Over time, better techniques were devised to virtualize
the x86 ISA, including ISA extensions (e.g., AMD-V and Intel VT-x) to enable
native execution of virtualized applications, only trapping to the hypervisor
when the application attempts to perform sensitive operations.

Several other forms of virtualization have also been considered.
Sun Microsystems popularized \emph{application virtualization} in the 1990s
with the Java programming language: applications are written to an abstract
machine---the \emph{Java Virtual Machine (JVM)}---which is backed by a runtime
system that ensures the program can execute on any platform.
This scheme eschews \emph{compatibility}---the ability to execute unmodified
legacy applications---for \emph{portability}.
\emph{Operating system-level virtualization} (e.g., Library OSes, Containers)
virtualizes yet another layer in the software stack: the operating system's
interfaces (e.g., system calls, kernel name-spaces). This style of
virtualization preserves compatibility by transparently modifying the
interfaces the application uses to access system resources, and results in
low overhead execution.

The proposed dissertation is primarily concerned with hardware virtualization.
Hardware virtualization is vital to high utilization of available physical
resources in large computing installations, e.g., hardware virtualization is
foundational to \textit{cloud computing}.
There have been many attempts to define hardware virtualization, from Popek
and Goldberg's classical virtualization properties---\emph{equivalence,
performance, and safety} to Bugnion, Nieh and
Tsafrir's~\cite{bugnion-nieh-tsafrir} definition of virtualization---%
\textit{``the application of the layering principle with
enforced modularity such that the exposed resource is identical to the
underlying resource''}. While technically correct, all of these defintions are
too contrite to be useful. Instead, for the purposes of this dissertation, we
concern ourselves with the following overarching goal---\emph{realizable,
fair, isolated, and efficient sharing of hardware resources among mutually
distrustful entities.}

Hardware virtualization typically involves mediating access to the shared
resource either by exposing an interface that is identical to that of the
physical resource (\emph{full-virtualization}), or by exposing an alternative
interface, operations on which are in-turn synthesized to the native interface
(\emph{para-virtualization}).
The exposed interface is \emph{virtual}, in that it is not directly exposed by
the physical underlying hardware, and instead is entirely under the control of
supervisory virtualization software, the \emph{hypervisor} (also known as the
\emph{Virtual Machine Monitor}). While operations in the resulting \emph{
virtual machine} may be directly executed on the physical hardware for
performance reasons, as in the case of hardware-assisted virtualization
schemes like AMD-V and Intel VT-x, all privileged operations still trap to the
hypervisor. The interface interposed may be a hardware interface (ISA, Memory,
I/O Protocols, etc.) or a software interface (Syscalls, APIs, etc.).

\subsection{CPU virtualization}

Four decades of attention from both the academic community and industry has
given rise to a large body of techniques that enable efficient virtualization
of CPUs: software techniques such as binary translation and device emulation,
are well established. While dominant ISAs, such as x86 and ARM, even provide
extensions to enable low-overhead virtualization, binary translation results
in lower overhead for sequences of sensitive instructions that need to be
emulated~\cite{vmware-esx-bt-plus-vtx}.

When implementing a new binary translator~\cite{HVX} or developing a
secure virtual instruction set, the developer is left to
their own devices to answer questions regarding \emph{realizability}, e.g., to
prioritize different parts of the ISA during development or to understand the
relative value of different parts of the ISA to backwards compatibility.
Predictably, developers have typically adopt ad-hoc methodologies to overcome
this challenge~\cite{bugnion-workstation}. We hypothesize that a principled
approach to answering these questions lies in understanding the distribution
of importance, to users, in the ISA being virtualized. Chapter 1 of the
dissertation will present a methodology for determining user preference, and
the resulting dataset.
Briefly, we estimate the importance of an instruction in the ISA by measuring
its frequency of occurrence in applications, and then weighting the frequency
data with the likelihood of users installing those applications. This is
completed work~\cite{x86-systor}.

\subsection{Accelerator virtualization}
Compute heavy and data parallel workloads such as graph processing and machine
learning have precipitated a Cambrian explosion of specialized processors.
These emerging compute devices (e.g., GPGPUs, TPUs, IPUs, IO accelerators),
however, pose a challenge to virtualization developers, who once again find
themselves balancing the essential characteristics of a virtualization
scheme---compatibility, interposition, sharing, isolation---with the need to
preserve the raw performance these processors provide. Virtualization
techniques developed for CPUs (ISA virtualization) are not applicable to these
specialized accelerators: their control interfaces are closer to those of I/O
devices than the ISAs of CPUs.
Techniques developed for I/O devices, such as NICs, are also untenable for
specialized compute devices as they result in the sacrifice of one or more of
the essential characteristics listed above.
Full-virtualization based schemes, such as GPUvm~\cite{suzuki2014gpuvm},
suffer from massive overheads that essentially negate the speedup that makes
the specialized compute unit attractive in the first place.
Para-virtual systems, such as SVGA~\cite{dowty2009gpu} that interpose on
low-level interfaces, such as the kernel driver, introduce much lower overhead
than full-virtualization based schemes but have poor compatibility, i.e., the
introduction of an artificial abstract interface constructed expressly for the
purpose of interposition necessitates massive engineering effort to support
new hardware in the host and new software frameworks in the guest.
User-space API-remoting solutions~\cite{vmCUDA,rCUDA,rCUDAnew} interpose on
the user-space API in the guest and forward the interposed operation to the
host as an RPC. This approach introduces very low overhead and can evolve with
the hardware easily, but has traditionally eschewed hypervisor interposition,
thereby making it difficult to enforce safety and isolation among guests.

Virtualizing a Graphics Processing Unit (GPU) for the purposes of graphics
rendering is a well studied problem, with existing commercial solutions, e.g.,
VMware’s SVGA~\cite{dowty2009gpu}. Over the last decade, GPUs have been
re-purposed for parallel general purpose compute (commonly known as GPGPU).
Chapter 2 of the proposed dissertation will present our findings from
attempting to extend the SVGA model of GPU virtualization to cover GPGPU
virtualization as well. We find that the tight coupling between ISA
virtualization and device virtualization in SVGA leads to poor performance for
GPGPU compute. We propose a new virtualization scheme, Trillium, that doesn’t
rely on ISA virtualization and show that Trillium outperforms all other
traditional virtualization schemes while retaining hypervisor interposition.
Material presented in Chapter 2 will be drawn from a published paper~\cite{trillium}.

Specialized compute units (e.g., Google TPU, Intel QAT, etc.) are
typically exposed to developers via a user-space API. The API is typically
implemented by a combination of proprietary software that interacts with the
hardware through opaque interfaces.
Chapters 3, 4 and 5 of the proposed dissertation will explore the performance
implications of virtualizing the user-space API for specialized compute
accelerators.
Chapter 3 will present an overview of AvA, a framework that enables automated
virtualization of accelerator APIs. Chapter 4 will focus on the performance
implications of API-remoting based virtualization of a single specialized
accelerator. Chapters 3 and 4 will draw on material that appeared in a HotOS
workshop paper~\cite{ava-hotos} and a full paper that is currently under
submission. Chapter 5 will explore performance issues that arise when an
application uses multiple API-remoted virtual accelerators in a pipelined
fashion, and is proposed work.

Virtualization schemes are traditionally taxonomized according to the core
techniques employed (e.g. emulation, full- or para-virtualization, API
remoting, etc.), and evaluated in a property trade-off space comprising
performance, compatibility, interposition, and isolation. We argue that both
the de facto taxonomy and the property trade-off space are illustrative but
not informative for GPGPU virtualization: there is a large body of research
that has had little influence on practice. We suggest an alternative framework
called \texttt{IEMTS} that teases apart design axes that are implicitly and
unnecessarily intertwined in much of the literature. By focusing on the
\textbf{I}nterface interposed, the interposition \textbf{E}ndpoints, the
\textbf{M}echanism of interposition, the \textbf{T}ransport used to move the
interposed operations between the guest and the host, and the mechanism used
to \textbf{S}ynthesize the interposed interface, \texttt{IEMTS} enables a
clearer understanding of trade-offs in prior designs and provides a model for
comparison of alternative designs. \texttt{IEMTS} will be presented in Chapter
6 of the proposed dissertation, along with analysis of traditional
virtualization techniques in the context of GPGPUs.

\noindent Concretely, the proposed dissertation will evaluate the following
hypotheses:
\begin{enumerate}[noitemsep, topsep=0pt, leftmargin=1em, labelwidth=*, align=left, label=\textbf{H \arabic*:}]
\item Development priority among instructions in an ISA, in the context of binary translation, can be automatically inferred from user preferences. (Dissertation Chapter 1)
\item ISA virtualization is unnecessary for performant virtualization of compute accelerators. (Dissertation Chapter 2)
\item Hypervisor-mediated API-remoting is a low-overhead virtualization scheme for API-controlled compute accelerators. (Dissertation Chapters 3, 4, and 5)
\item A virtualization technique can be succinctly characterized by a scheme
that explicitly captures the \textit{Interface} interposed,
the \textit{Endpoints} interposed on, the \textit{Mechanism} of
interposition, the \textit{Transport} used to connect the interposed
endpoints, and the mechanism used to \textit{Synthesize} the interposed
operation on the host. (Dissertation Chapter 6)
\end{enumerate}

The proposed dissertation will draw from prior work that was carried out as a
team effort, as is common when building large systems.