% !TeX root = proposal.tex
\section{Introduction}
\label{sec:intro}

Virtualization, defined broadly, is a means to an end: fair, isolated, and
efficient sharing of resources among mutually distrustful entities.
Virtualization is vital to achieving high utilization of available physical
resources in large computing installations such as clusters and data centers.
Virtualization is the main technical force behind cloud computing.

Virtualizing a compute resource, such as a CPU, typically involves mediating
access to said resource either by exposing an interface that is identical to
that of the encapsulated resource (full-virtualization), or by
exposing an alternative abstract interface, accesses to which are in-turn
synthesized to the native interface (para-virtualization).
The exposed interface is \emph{virtual}, in that it is not directly exposed by
the physical underlying hardware, and instead is entirely under the control of
supervisory virtualization software, the \emph{hypervisor} (also known as the \emph{Virtual Machine Monitor}). The hypervisor is responsible for ensuring

Despite nearly four decades of attention from both the academic community and
industry, efficient virtualization of compute resources remains poorly
understood. As new compute devices emerge (e.g., GPGPUs, TPUs, IPUs, IO
accelerators), virtualization developers find themselves once again balancing
the essential characteristics of a good virtualization scheme---compatibility,
interposition, sharing, isolation---with the need to preserve the raw
performance these emerging compute resources provide.

\noindent Concretely, the proposed dissertation will evaluate the following
hypotheses:
\begin{itemize}[noitemsep,topsep=0pt,leftmargin=1em,labelwidth=*,align=left]
\item Priority among instructions in an ISA, in the context of binary
translation, can be automatically inferred from user preferences.
\item ISA virtualization is untenable for performant virtualization of compute
accelerators.
\item Hypervisor-mediated API-remoting is a low-overhead virtualization scheme
for API-controlled compute accelerators.
\item The characteristics of a virtualization technique can be succinctly
described by a scheme that explicitly captures the \textit{Interface}
interposed, the \textit{Endpoints} interposed on, the \textit{Mechanism} of
interposition, the \textit{Transport} used to connect the interposed
endpoints, and the mechanism used to \textit{Synthesize} the interposed
operation on the host.
\end{itemize}