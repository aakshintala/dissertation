% !TeX root = proposal.tex
\section{Related Work}
\label{sec:related}

Virtualization has such a long and storied history that Attempting to capture
the entire story is an exercise in futility. The introduction~\ref{sec:intro}
captures the history of CPU virtualization in broad strokes. This section then
focuses on a major theme of the proposed dissertation:
accelerator virtualization.

Accelerating specific computation is not a new idea---support for specialized
computation is extremely commonplace in CPUs (e.g., Floating Point Units
(FPU), Vector Processing Units). These specialized compute units are
typically exposed to the programmer as extensions to the Instruction Set
Architecture (ISA). Virtualizing these specialized compute units, therefore,
is no different from virtualizing the rest of the CPU and ISA virtulaization
is well explored~\cite{cp40,vm370,popek-goldberg,bugnion-disco,
bugnion-nieh-tsafrir,bugnion-workstation}.

Processors specialized for complex computational tasks, such as graphical
rendering, largely evolved as discrete devices separate from the CPU (although
some CPUs do integrate GPUs). These devices are not typically integrated into
the CPU ISA; instead, they appear to system software as I/O devices with
memory-mapped command-queues and I/O registers. I/O virtualization is well
understood~\cite{waldspurger12cacm,paradice,Kuperman_undated-io,Sig2010-ml,
zeng2013improved,abramson2006intel}, but


% !TeX root = proposal.tex
\def\libnomod{lib unmod\xspace}
\def\osnomod{OS unmod\xspace}
\def\sharing{sharing\xspace}
\def\isolation{isolation\xspace}
\def\scheduling{sched. policy\xspace}
\def\perf{performance\xspace}
\def\libflex{lib-compat\xspace}
\def\hwflex{hw-compat\xspace}
\def\mobility{migration\xspace}
\def\gfx{graphics\xspace}
\def\gpgpu{GPGPU\xspace}
\def\id{I/D}
\def\discrete{\emph{D}}
\def\integrated{\emph{I}}
% \def\poor{--}
% \def\good{+}
% \def\ok{+/-}
% \def\verygood{++}
% \def\verybad{-- --}
\def\poor{\textbf{poor}\xspace}
\def\good{\textbf{good}\xspace}
\def\ok{\textbf{ok}\xspace}
\def\verygood{\textbf{excellent}\xspace}
\def\verybad{\textbf{bad}\xspace}
%\def\chk{$\times$}
\def\chk{\checkmark}
\def\cross{$\times$\xspace}
\def\gr{\cellcolor[gray]{0.9}}
\def\redc{\cellcolor[gray]{0.4}}
\def\bluec{\cellcolor[gray]{0.1}}

%\newcommand{\mc}[2]{\multicolumn{#1}{c}{#2}}
\definecolor{Gray}{gray}{0.6}
\definecolor{LightCyan}{rgb}{0.88,1,1}
%\newcolumntype{a}{>{\columncolor{Gray}}c}
\def\FV{\textbf{FV}}
\def\PT{\textbf{PT}}
\def\PV{\textbf{PV}}
\def\APIR{\textbf{API-R}}
\def\unmodlib{\textbf{UL}}
\def\unmodos{\textbf{UOS}}
\def\safe{\textbf{I}}
\def\fair{\textbf{F}}
\def\mob{\textbf{M}}

\setlength{\aboverulesep}{0pt}
\setlength{\belowrulesep}{0pt}

\begin{table*}[ht!]
\vspace*{2em}
\centering
\footnotesize{
\resizebox{\textwidth}{!}{
\begin{tabular}{r|l|c|c|c|c|c|c|c|c|c|c|c|c|c|c|c|}
%\hline
\T\B {\textbf{Technique}}       &
\T\B {\textbf{System}}          &
\rot{\textbf{\libnomod}}        &
\rot{\textbf{\osnomod}}         &
\rot{\textbf{\libflex}}         &
\rot{\textbf{\hwflex}}          &
\rot{\textbf{\sharing}}         &
\rot{\textbf{\isolation}}       &
\rot{\textbf{\mobility}}        &
\rot{\textbf{\parbox{4cm}{sched. \\policy}\xspace}}      &
\rot{\textbf{\gfx}}             &
\rot{\textbf{\gpgpu}}           &
\rot{\textbf{\id}}              &
\rot{\textbf{benchmark}}        &
\rot{\textbf{slowdown}}         &
\rot{\textbf{\parbox{4cm}{native\\speedup}}}    &
\rot{\textbf{\parbox{4cm}{virtual\\speedup}}}
\\   \hline

\T\B \multirow{2}{*}{\textbf{Full-virtual}}     &
     \T\B \textbf{GPUvm~\cite{GPUvm}}           &
     \T\B \chk                                  &  % unmodified guest libraries
     \T\B                                       &  % unmodified guest OS
     \T\B \chk                                  &  % lib-compatibility
     \T\B                                       &  % hw-compatibility
     \T\B \chk                                  &  % cross-VM sharing
     \T\B \chk                                  &  % cross-VM isolation
                                                &  % VM migration
     \T\B XC, BAND                              &  % fairness
                                                &  % graphics support
     \T\B \chk                                  &  % GPGPU support
     \T\B \discrete                             &  % integrated-discrete
     \T\B Rodinia                               &  % benchmark
     \T\B 141$\times$                           &  % base slowdown
     \T\B 11.4$\times$                          &  % native speedup
     \T\B {\textcolor{red}{0.08$\times$}}          % virtualized speedup
     \\ \cline{2-17}
                                                &
     \T\B \textbf{gVirt~\cite{gVirt}}           &
     \T\B \chk                                  &  % unmodified guest libraries
     \T\B                                       &  % unmodified guest OS
     \T\B \chk                                  &  % compatibility
     \T\B                                       &  % compatibility
     \T\B \chk                                  &  % cross-VM sharing
     \T\B \chk                                  &  % cross-VM isolation
     \T\B \chk                                  &  % VM migration
     \T\B QoS                                   &  % fairness
     \T\B \chk                                  &  % graphics support
     \T\B                                       &  % GPGPU support
     \T\B \integrated                           &  % integrated-discrete
     \T\B 2D~\cite{phoronix}, 3D~\cite{cairoperf}& % benchmark
     \T\B 1.6$\times$                           &  % base slowdown
     \T\B N/A                                   &  % native speedup
     \T\B N/A                                      % virtualized speedup
     \\ \hline

\T\B \textbf{PCIe Pass-thru}                    &
     \T\B \textbf{AWS GPU~\cite{amazongpu}}     &
     \T\B \chk                                  & % unmodified guest lib
     \T\B \chk                                  & % unmodified guest OS
     \T\B \cellcolor{gray!25}                   & % compatibility
     \T\B \cellcolor{gray!25}                   & % compatibility
     \T\B \cellcolor{gray!25}                   & % sharing
     \T\B \cellcolor{gray!25}                   & % isolation
     \T\B \cellcolor{gray!25}                   & % mobility
     \T\B \cellcolor{gray!25}                   & % fairness
     \T\B \chk                                  & % graphics
     \T\B \chk                                  &  % GPGPU support
     \T\B \discrete                             &  % integrated-discrete
     \T\B Any                                   &  % benchmark
     \T\B 1$\times$                             &  % base slowdown
     \T\B \cellcolor{gray!25}                   &  % natice speedup
     \T\B \cellcolor{gray!25}                      % virtualized speedup
     \\ \hline

\T\B \multirow{4}{*}{\bf API remoting}          &
     \T\B {\bf GViM~\cite{gupta2009gvim}}       &
     \T\B                                       &  % unmodified guest libraries
     \T\B                                       &  % unmodified guest OS
     \T\B                                       &  % lib-compatibility
     \T\B \chk                                  &  % hw-compatibility
     \T\B \chk                                  &  % cross-VM sharing
     \T\B \chk                                  &  % cross-VM isolation
                                                &  % VM migration
     \T\B RR, XC                                &  % fairness
                                                &  % graphics support
     \T\B \chk                                  &  % GPGPU support
     \T\B \discrete                             &  % integrated-discrete
     \T\B CUDA 1.1 SDK                          &  % benchmark
     \T\B 1.16$\times$                          &  % base slowdown
     \T\B 22$\times$                            &  % native speedup
     \T\B {\textcolor{blue}{19$\times$}}           % virtualized speedup
     \\ \cline{2-17}


                                                &
     \T\B {\bf gVirtuS~\cite{gVirtuS}}          &
     \T\B                                       &  % unmodified guest libraries
     \T\B                                       &  % unmodified guest OS
     \T\B                                       &  % lib-compatibility
     \T\B \chk                                  &  % hw-compatibility
     \T\B \chk                                  &  % cross-VM sharing
     \T\B \chk                                  &  % cross-VM isolation
                                                &  % VM migration
     \T\B RR                                    &  % fairness
                                                &  % graphics support
     \T\B \chk                                  &  % GPGPU support
     \T\B \discrete                             &  % integrated-discrete
     \T\B CUDA 2.3 MM                           &  % benchmark
     \T\B 3.1$\times$                           &  % base slowdown
     \T\B 11.1$\times$                          &  % native speedup
%     \T\B 3.6$\times$                              % virtualized speedup
     \T\B {\textcolor{blue}{3.6$\times$}}          % virtualized speedup
     \\ \cline{2-17}

                                                &
     \T\B \T\B {\bf vCUDA~\cite{shi2012vcuda}}         &
     \T\B                                       &  % unmodified guest libraries
     \T\B \chk                                  &  % unmodified guest OS
     \T\B                                       &  % lib-compatibility
     \T\B \chk                                  &  % hw-compatibility
     \T\B                                       &  % cross-VM sharing
     \T\B                                       &  % cross-VM isolation
     \T\B \chk                                  &  % VM migration
     \T\B HW                                    &  % fairness
                                                &  % graphics support
     \T\B \chk                                  &  % GPGPU support
     \T\B \discrete                             &  % integrated-discrete
     \T\B CUDA 4.0 SDK                          &  % benchmark
     \T\B 1.91$\times$                          &  % base slowdown
     \T\B 6$\times$                            &  % native speedup
     \T\B {\textcolor{blue}{3.1$\times$}}          % virtualized speedup
     \\ \cline{2-17}

                                                &
     \T\B {\bf vmCUDA~\cite{vmCUDA}}            &
     \T\B                                       &  % unmodified guest libraries
     \T\B \chk                                  &  % unmodified guest OS
     \T\B                                       &  % lib-compatibility
     \T\B \chk                                  &  % hw-compatibility
     \T\B \chk                                  &  % cross-VM sharing
     \T\B                                       &  % cross-VM isolation
     \T\B                                       &  % VM migration
     \T\B HW                                    &  % fairness
                                                &  % graphics support
     \T\B \chk                                  &  % GPGPU support
     \T\B \discrete                             &  % integrated-discrete
     \T\B CUDA 5.0 SDK                          &  % benchmark
     \T\B 1.04$\times$                          &  % base slowdown
     \T\B 33$\times$                            &  % native speedup
     \T\B {\textcolor{blue}{31.7$\times$}}         % virtualized speedup
     \\ \hline


\T\B \multirow{4}{*}{\bf \shortstack[r]{Distributed\\API remoting}}  &
    \T\B {\bf rCUDA~\cite{rCUDA, rCUDAnew}}     &
     \T\B                                       &  % unmodified guest libraries
     \T\B \chk                                  &  % unmodified guest OS
     \T\B                                       &  % lib-compatibility
     \T\B \chk                                  &  % hw-compatibility
     \T\B \chk                                  &  % cross-VM sharing
     \T\B \chk                                  &  % cross-VM isolation
     \T\B                                       &  % VM migration
     \T\B RR                                    &  % fairness
                                                &  % graphics support
     \T\B \chk                                  &  % GPGPU support
     \T\B \discrete                             &  % integrated-discrete
     \T\B CUDA 3.1 SDK                          &  % benchmark
     \T\B 1.83$\times$                          &  % base slowdown
     \T\B 49.8$\times$                          &  % native speedup
     \T\B {\textcolor{blue}{27.2$\times$}}         % virtualized speedup
     \\ \cmidrule{2-17}

                                                &
    \T\B {\bf GridCuda~\cite{GridCuda}}         &
     \T\B                                       &  % unmodified guest libraries
     \T\B \chk                                  &  % unmodified guest OS
     \T\B                                       &  % lib-compatibility
     \T\B \chk                                  &  % hw-compatibility
     \T\B \chk                                  &  % cross-VM sharing
     \T\B \chk                                  &  % cross-VM isolation
     \T\B                                       &  % VM migration
     \T\B FIFO                                  &  % fairness
                                                &  % graphics support
     \T\B \chk                                  &  % GPGPU support
     \T\B \discrete                             &  % integrated-discrete
     \T\B CUDA MM, SOR                          &  % benchmark
     \T\B 1.23$\times$                          &  % base slowdown
     \T\B \cellcolor{gray!10}                   &  % native speedup
     \T\B \cellcolor{gray!10}                      % virtualized speedup
     \\ \cmidrule{2-17}

                                                &
    \T\B {\bf SnuCL~\cite{kim2012snucl}}        &
     \T\B                                       &  % unmodified guest libraries
     \T\B \chk                                  &  % unmodified guest OS
     \T\B                                       &  % compatibility
     \T\B                                       &  % compatibility
     \T\B \chk                                  &  % cross-VM sharing
     \T\B \chk                                  &  % cross-VM isolation
     \T\B                                       &  % VM migration
     \T\B \cellcolor{gray!25}                   &  % fairness
                                                &  % graphics support
     \T\B \chk                                  &  % GPGPU support
     \T\B \discrete                             &  % integrated-discrete
     \T\B SNU NPB~\cite{seo2011performance}     &  % benchmark
     \T\B \cellcolor{gray!25}                   &  % base slowdown
     \T\B \cellcolor{gray!10}                   &  % native speedup
     \T\B \cellcolor{gray!25}                      % virtualized speedup
     \\ \cmidrule{2-17}

                                                &
    \T\B {\bf VCL~\cite{VCL}}                   &
     \T\B                                       &  % unmodified guest libraries
     \T\B \chk                                  &  % unmodified guest OS
     \T\B                                       &  % lib-compatibility
     \T\B \chk                                  &  % hw-compatibility
     \T\B \chk                                  &  % cross-VM sharing
     \T\B \chk                                  &  % cross-VM isolation
     \T\B                                       &  % VM migration
     \T\B \cellcolor{gray!25}                   &  % fairness
                                                &  % graphics support
     \T\B \chk                                  &  % GPGPU support
     \T\B \discrete                             &  % integrated-discrete
     \T\B Stencil2D~\cite{danalis2010scalable}  &  % benchmark
     \T\B \cellcolor{gray!25}                   &  % base slowdown
     \T\B \cellcolor{gray!10}                   &  % native speedup
     \T\B \cellcolor{gray!25}                      % virtualized speedup
%     \T\B {\textcolor{blue}{3.4$\times$}}          % virtualized speedup
     \\ \hline

\T\B \multirow{7}{*}{\textbf{Para-virtual}} &
     \T\B \textbf{GPUvm~\cite{GPUvm}}           &
     \T\B                                       &  % unmodified guest libraries
     \T\B                                       &  % unmodified guest OS
     \T\B                                       &  % compatibility
     \T\B                                       &  % compatibility
     \T\B \chk                                  &  % cross-VM sharing
     \T\B \chk                                  &  % cross-VM isolation
                                                &  % VM migration
     \T\B XC, BAND                              &  % fairness
                                                &  % graphics support
     \T\B \chk                                  &  % GPGPU support
     \T\B \discrete                             &  % integrated-discrete
     \T\B Rodinia                               &  % benchmark
     \T\B 5.9$\times$                           &  % base slowdown
     \T\B 11.4$\times$                          &  % native speedup
     \T\B {\textcolor{blue}{1.9$\times$}}          % virtualized speedup
     \\ \cmidrule{2-17}

                                               &
    \T\B {\bf HSA-KVM~\cite{kaveri16vee}}      &
    \T\B \chk                                  &  % unmodified guest libraries
    \T\B                                       &  % unmodified guest OS
    \T\B                                       &  % compatibility
    \T\B                                       &  % compatibility
    \T\B \chk                                  &  % cross-VM sharing
    \T\B \chk                                  &  % cross-VM isolation
                                               &  % VM migration
    \T\B HW                                    &  % fairness
    &  % graphics support
    \T\B \chk                                  &  % GPGPU support
    \T\B \integrated                           &  % integrated-discrete
    \T\B AMD OCL SDK                           &  % benchmark
    \T\B 1.1$\times$                           &  % base slowdown
    \T\B \cellcolor{gray!10}                   &  % native speedup
    \T\B \cellcolor{gray!10}                      % virtualized speedup
    \\ \cline{2-17}
                                                &
     \T\B {\bf LoGV~\cite{logv}}                &
     \T\B \chk                                  &  % unmodified guest libraries
     \T\B                                       &  % unmodified guest OS
     \T\B \chk                                  &  % compatibility
     \T\B                                       &  % compatibility
     \T\B \chk                                  &  % cross-VM sharing
     \T\B \chk                                  &  % cross-VM isolation
     \T\B \chk                                  &  % VM migration
     \T\B RR                                    &  % fairness
     \T\B                                       &  % graphics support
     \T\B \chk                                  &  % GPGPU support
     \T\B \discrete                             &  % integrated-discrete
     \T\B Rodinia                               &  % benchmark
     \T\B 1.01$\times$                          &  % base slowdown
     \T\B 11.4$\times$                          &  % native speedup
     \T\B {\textcolor{blue}{11.3$\times$}}         % virtualized speedup
     \\ \cmidrule{2-17}

                                                &
     \T\B \textbf{SVGA2~\cite{dowty2009gpu}} &
     \T\B \chk                                  &  % unmodified guest libraries
     \T\B                                       &  % unmodified guest OS
     \T\B                                       &  % compatibility
     \T\B                                       &  % compatibility
     \T\B \chk                                  &  % cross-VM sharing
     \T\B \chk                                  &  % cross-VM isolation
     \T\B \chk                                  &  % VM migration
     \T\B \cellcolor{gray!25}                   &  % fairness
     \T\B \chk                                  &  % graphics support
                                                &  % GPGPU support
     \T\B \discrete                             &  % integrated-discrete
     \T\B 2D, gaming                            &  % benchmark
     \T\B 3.9$\times$                           &  % base slowdown
     \T\B \cellcolor{gray!25}                   &  % native speedup
     \T\B \cellcolor{gray!25}                      % virtualized speedup
     \\ \cmidrule{2-17}

                                                &
     \T\B \textbf{Paradice~\cite{paradice}}     &
     \T\B \chk                                  &  % unmodified guest libraries
     \T\B                                       &  % unmodified guest OS
     \T\B \chk                                  &  % compatibility
     \T\B                                       &  % compatibility
     \T\B \chk                                  &  % cross-VM sharing
     \T\B \chk                                  &  % cross-VM isolation
                                                &  % VM migration
     \T\B HW, QoS                               &  % fairness
     \T\B \chk                                  &  % graphics support
     \T\B \chk                                  &  % GPGPU support
     \T\B \discrete                             &  % integrated-discrete
     \T\B OpenGL, OpenCL                        &  % benchmark
     \T\B 1.1$\times$                           &  % base slowdown
     \T\B \cellcolor{gray!10}                   &  % native speedup
     \T\B \cellcolor{gray!10}                      % virtualized speedup
     \\ \cmidrule{2-17}

                                                &
     \T\B \textbf{VGVM~\cite{vasila-gvm16}}     &
     \T\B                                       &  % unmodified guest libraries
     \T\B                                       &  % unmodified guest OS
     \T\B                                       &  % compatibility
     \T\B \chk                                  &  % compatibility
     \T\B \chk                                  &  % cross-VM sharing
     \T\B \chk                                  &  % cross-VM isolation
                                                &  % VM migration
     \T\B HW                                    &  % fairness
                                                &  % graphics support
     \T\B \chk                                  &  % GPGPU support
     \T\B \discrete                             &  % integrated-discrete
     \T\B CUDA 5.0 SDK                          &  % benchmark
     \T\B 1.02$\times$                          &  % base slowdown
     \T\B 33$\times$                            &  % native speedup
     \T\B {\textcolor{blue}{32.3$\times$}}         % virtualized speedup
     \\ \hline

\end{tabular}
}
}
\caption{Existing GPU virtualization proposals, grouped by approach. Previously published in the Trillium paper~\cite{trillium}.}
\label{tab:virt-comp}
% \vspace*{-6pt}
\end{table*}

GPU virtualization has received a lot of attention since the late 2000s. This
section presents an overview of all prior work.
Table~\ref{tab:virt-comp} presents a comprehensive overview of prior
accelerator virtualization techniques in terms of traditional virtualization
properties. The \textbf{\libnomod} and \textbf{\osnomod}
columns indicate ability to support unmodified guest libraries and OS/driver.
The \textbf{\libflex} and \textbf{\hwflex} indicate the ability
(compatibility) to support a GPU device abstraction that is independent of
\textit{framework} or \textit{hardware} actually present on the host.
\textbf{\sharing}, \textbf{\isolation}
and \textbf{\scheduling} indicate cross-domain sharing, isolation and
some attempt to support fairness or performance isolation
(policies such as RR Round-Robin, XC XenoCredit, HW hardware-managed, etc.).
The \textbf{\mobility} shows support for VM migration.
\textbf{I/D} indicates it supports either integrated or discrete GPU. The
table also includes performance entries for each system including the
geometric-mean slowdown (execution time relative to native execution) across
all reported benchmarks. We additionally include the benchmarks used, and
where possible, a report (or estimate) of the geometric-mean speedup one
should \emph{expect} for using GPUs over CPUs using hardware similar to that
used in this paper. The final column is the expected geometric-mean speedup
for the given benchmarks running in the virtual GPGPU system over running on
native CPUs. Values in this column were computed by dividing the expected
speedup from using a GPU by the slowdown introduced by virtualization.
Entries where overheads eclipse GPU-based performance gains are marked in
\textcolor{red}{red}. Performance profitable entries are
\textcolor{blue}{blue}. \textcolor{darkgray}{Greyed out cells} indicate the
metric is meaningless for that design. \textcolor{gray}{Light grey cells}
indicate that the data was not available.





\aak{Discuss AmorphOS, HPVM, HSA-KVM, PTask, Dandelion, TornadoVM. Add details about the GPU virtualization stuff in the table above. }